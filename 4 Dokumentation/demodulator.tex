%\title{18W MOSFET amplifier)
% Title: 18W MOSFET amplifier, with npn transistor.
% Author: Ramón Jaramillo.
% Source: http://www.circuitstoday.com/mosfet-amplifier-circuits
% Abstract:
% This example made use of circuitikz and siunits packages for drawing a 18W MOSFET  Amplifier for one-channel. It is mandatory than siunitx and related packages have been instaled, acording to your LaTeX distribution.
\documentclass[tikz,border=10pt
              ]{standalone}
\usepackage{tikz}
\usetikzlibrary{shapes,arrows,automata,matrix,trees}
\usetikzlibrary{decorations.pathreplacing,calc,positioning}
\usetikzlibrary{fit}					% fitting shapes to coordinates
\usetikzlibrary{dsp,chains}
\usepackage{tikz-timing}[2009/12/09]
\usetikztiminglibrary[new={char=Q,reset char=R}]{counters}
\usetikzlibrary{decorations,arrows} 
\usetikzlibrary{decorations.pathmorphing} 
\usepgflibrary{decorations.pathreplacing} 
\usetikzlibrary{dsp,chains}
\begin{document}
\begin{tikzpicture}
	\matrix (m0) [row sep=2.5mm, column sep=12mm]
	{	%--------------------------------------------------------------------
		\node[coordinate]                  (m00) {};   		  &
		\node[coordinate]                  (m01) {};   		  &
		\node[dspmixer, pin={[pin edge={to-,very thick,black}]below:$cos(2\pi f_0)$}]                    (m02) {};   		  &
		\node[dspsquare]                   (m03) {LPF};   	  &
		\node[dspnodeopen,dsp/label=below] (m0X) {$I(t)$};	  \\				%--------------------------------------------------------------------
		\node[dspnodeopen,dsp/label=above] (m10) {$f_{RF}(t)$};    &
		\node[dspnodefull]                 (m11) {};          &
		\node[coordinate]                  (m12) {};          &
		\node[coordinate]                  (m13) {};          &
		\node[coordinate]                  (m1X) {};          \\		%--------------------------------------------------------------------
		\\		%--------------------------------------------------------------------
		\node[coordinate]                  (m20) {};          &
		\node[coordinate]                  (m21) {};          &
		\node[dspmixer, pin={[pin edge={to-,very thick,black}]below:$sin(2\pi f_0)$}]                    (m22) {};   		  &
		\node[dspsquare]                   (m23) {LPF};		  &
		\node[dspnodeopen,dsp/label=below] (m2X) {$Q(t)$};    \\		%--------------------------------------------------------------------		%--------------------------------------------------------------------
		\node[coordinate] (m30) {}; &
		\node[coordinate] (m31) {}; &
		\node[coordinate] (m32) {}; &
		\node[coordinate] (m33) {}; &
		\node[coordinate] (m3X) {};    \\		%--------------------------------------------------------------------
	};	
	\begin{scope}[start chain]
		\chainin (m10);
		\chainin (m11) [join=by dspconn];
		\chainin (m01) [join=by dspline];
		\chainin (m21) [join=by dspline];
	\end{scope}
	%\draw[dspconn]  (m12) --  (m02);	
	%\draw[dspconn]  (m32) -- node[below] {$sin(2\pi f_0)$} (m22);	
			
	\foreach \i [evaluate = \i as \j using int(\i+1)] in {1,..., 2}
	{
		\draw[dspconn] (m0\i) -- (m0\j);
		\draw[dspconn] (m2\i) -- (m2\j);
	}
	\draw[dspflow] (m03) -- (m0X);
	\draw[dspflow] (m23) -- (m2X);
\end{tikzpicture}
\end{document}