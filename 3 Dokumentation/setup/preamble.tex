%\usepackage{settings}
\usepackage{amsmath,amstext,amsfonts,amssymb} %Mathe-Formeln

\usepackage{graphicx} %Einbinden von extern erzeugten Grafiken
\graphicspath{{images/}} %Pfad zu den Bildern (einfache Referenzierung)
\usepackage{epstopdf}
\usepackage{makeidx}

\usepackage{pdflscape} %Change orientation of specific part
\usepackage{afterpage}
\usepackage{rotating}

\usepackage[usenames,dvipsnames]{xcolor}
\usepackage{fancybox}
\usepackage{ulem} %für Text linien
\usepackage[utf8]{inputenc}
\usepackage[ngerman]{babel}
\usepackage[OT1]{fontenc} % Ligaturen, richtige Umlaute im PDF
%Umschalten der Sprache
\selectlanguage{ngerman}
\usepackage{marvosym}
\DeclareInputText{128}{\EUR}
%Anzeige mathematischer Symbole
\usepackage{amssymb}
%Anzeige von mathematischer Umgebung
\usepackage{amsthm}
\usepackage{stmaryrd}
\usepackage{verbatim}

%f?r integralzeichen Hoch- und Tiefstellung
\usepackage[intlimits]{empheq}
%PDF-Dokumente einbinden
\usepackage{pdfpages}
%Tabellen einbinden
%\usepackage{booktabs}
\usepackage{tabularx}
%\usepackage{tabular}
\usepackage{hhline}
\usepackage{enumerate}
\usepackage[format = hang, font = {footnotesize, sf}, labelfont = {bf}, margin = 1cm, aboveskip = 5pt, position = bottom]{caption}

%Seitenlayout
\usepackage[left = 2.5cm, right = 2.5cm, top = 1.5cm, bottom = 1.5cm, headheight = 7mm, headsep = 1.2 cm, footskip= 1.5cm, includeheadfoot]{geometry}

%Seitenstil = inhaltliche Gestaltung der Kopf- und Fu?zeile
%\pagestyle{headings}



\usepackage{fancyhdr}
\pagestyle{fancy}
	\fancyhf{}

	\fancyhead[L]{Entwicklung einer Dopplerinstrumentierung zur Detektion von Luftembolien in\\ einem künstlichen Blutkreislauf}
	\fancyhead[R]{\includegraphics[width=2cm]{images/HS_logo_sw.jpg}}
	\fancyfoot[R]{\thepage}
	\fancyfoot[L]{\nouppercase{\leftmark}}
\fancypagestyle{plain}
{
	\fancyhf{}
		
		\fancyhead[L]{Entwicklung einer Dopplerinstrumentierung zur Detektion von Luftembolien in\\ einem künstlichen Blutkreislauf}
		\fancyhead[R]{\includegraphics[width=2cm]{images/HS_logo_sw.jpg}}
			\fancyfoot[R]{\thepage}
	\fancyfoot[L]{\nouppercase{\leftmark}}
}
\fancypagestyle{number}
{
	\fancyhf{}
		\fancyfoot[R]{\thepage}		
}

%Zeilenabstand festlegen
\linespread{1.3}


%\newcommand{\HRule}{\rule{\linewidth}{0.5mm}}
%Nullen der Seitenzahl
%\setcounter{page}{0}

%Formelnummerierung anhand der Kapitel
\def\theequation{\thesection.\arabic{equation}}

%Paket zum zitieren
\usepackage{cite}

\usepackage{colortbl} % Um die Zellen innerhalb von \multicolumn farbig darzustellen, muss das Paket colortbl eingebunden werden

%Paket um sprachbezogene Anf?hrungszeichen zu setzen
%\usepackage{csquotes}

%EInbinden des Bibstyles
%\usepackage[backend = biber, style= numeric-comp, sorting = nyt, maxnames = 4, minnames = 1, abbreviate = false]{biblatex}

\newenvironment{myitemize}{\begin{itemize}\itemsep -4pt}{\end{itemize}}

\usepackage{longtable}
%Packet zum Einbinden von Graphiken mit umflie?endem Text
\usepackage{wrapfig}
%Packet zum Benutzen von mehrzeiligen Spalten
\usepackage{multirow}

%Striche raus aus Literaturverzeichnis ????? JM
%\makeatletter
%\def\bstctlcite{\@ifnextchar[{\@bstctlcite}{\@bstctlcite[@auxout]}}
%\def\@bstctlcite[#1]#2{\@bsphack
%  \@for\@citeb:=#2\do{%
%    \edef\@citeb{\expandafter\@firstofone\@citeb}%
%    \if@filesw\immediate\write\csname #1\endcsname{\string\citation{\@citeb}}\fi}%
%  \@esphack}
%\makeatother

%Kapitel etwas weiter nach oben setzen
\renewcommand*\chapterheadstartvskip{\vspace*{-0.2cm}}

\usepackage{indentfirst}

\usepackage{url}

%%
%% Trennungsregeln
%%
\hyphenation{Sil-ben-trenn-ung}

%%
%% Schönere Bullets bei Aufzählungen
%%
\renewcommand{\labelitemi}{$\bullet$}
%\renewcommand{\labelitemii}{$\cdot$}
\renewcommand{\labelitemiii}{$\cdot$}
\renewcommand{\labelitemiv}{$\ast$}
\usepackage{acronym}
%Bilder und Tabellen nebeneinander
\usepackage{caption}
\usepackage{subcaption}

\usepackage{tikz}
\usetikzlibrary{shapes,arrows,automata,matrix,trees}
\usetikzlibrary{decorations.pathreplacing,calc,positioning}
\usetikzlibrary{fit}					% fitting shapes to coordinates
\usepackage{footnote}
\usepackage{tikz-timing}[2009/12/09]
% Use tikz-timing library 'counters' to define a counter character.
% This character draws a 'D{<counter value>}' and increases the counter
% value by one. A reset character which resets the counter value 
% (by default to 1) is also defined.
\usetikztiminglibrary[new={char=Q,reset char=R}]{counters}
%\usepackage[active,tightpage]{preview}
%\PreviewEnvironment{tikzpicture}
\usetikzlibrary{dsp,chains}

\usepackage{ragged2e}

\DeclareMathAlphabet{\mathpzc}{OT1}{pzc}{m}{it}
\usepackage{amssymb}
\usepackage{eurosym} %für euronen..
\usepackage{paralist} %lässt sich die Art der Aufzählung ändern
\usetikzlibrary{plotmarks}
\usepackage[active,tightpage]{preview}
\newcommand{\dotprod}{{\scriptscriptstyle \stackrel{\bullet}{{}}}}

% \wraptextureframe{Höhe_in_Zeilen}{Breite}{Text}
% #1 Breite (Anzahl der Zeilen) und variable Höhe
% #2 Text
\newcommand{\wraptextureframe}[2][]{%
  \begin{wrapfigure}{r}{#1}
    \begin{Sbox}
      \begin{minipage}{#1-1em}
        \small\slshape#2%
      \end{minipage}
    \end{Sbox}
    \vskip-2ex%
    \fcolorbox{blue}{gray!0}{\TheSbox}
  \end{wrapfigure}%
}
\usepackage[,european]{circuitikz}
\newcommand*{\myMarvosymbol}[1]{{\fontfamily{mvs}\fontencoding{U}\fontseries{m}\fontshape{n}\selectfont\char#1}}
\newcommand*{\emailsymbol}{\myMarvosymbol{66}~}